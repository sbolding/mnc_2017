\documentclass[10pt,twocolumn,pdftex,superscriptaddress]{anstrans}
\usepackage{amsmath}
\usepackage{algorithm}
\usepackage{algorithmic}
\usepackage{txfonts}
\usepackage{multicol}
\usepackage[numbers,square]{natbib}
\usepackage{graphicx}
\usepackage{tabls}
\usepackage{multirow}
\usepackage{float}
\usepackage{widetext}
\usepackage{comment}
\usepackage{xcolor}
\graphicspath{{Figures/}}

\usepackage{hyperref}
\hypersetup{
linktocpage=true,
colorlinks=false,
bookmarks=true,
pdfborder={0 0 0},
citecolor=black,
urlcolor=black,
linkcolor=black,
citebordercolor=white,
urlbordercolor=white,
linkbordercolor=white,
breaklinks=true,
pdfpagelabels=true,
pageanchor=true
}

\renewcommand{\eqref}[1]{(\ref{#1})}
\renewcommand*\thetable{\Roman{table}}
\usepackage{caption}
\captionsetup[table]{labelsep=period}
\captionsetup[figure]{labelsep=period}
\newcommand{\N}{\mathbb{N}}
\newcommand{\Z}{\mathbb{Z}}
\newcommand{\deriv}[2]{\frac{\mathrm{d} #1}{\mathrm{d} #2}}
\newcommand{\pderiv}[2]{\frac{\partial #1}{\partial #2}}
\newcommand{\bx}{\mathbf{X}}
\newcommand{\ba}{\mathbf{A}}
\newcommand{\by}{\mathbf{Y}}
\newcommand{\bj}{\mathbf{J}}
\newcommand{\bs}{\mathbf{s}}
\newcommand{\B}[1]{\ensuremath{\mathbf{#1}}}
\newcommand{\Dt}{\Delta t}
\renewcommand{\d}{\mathrm{d}}
\newcommand{\mom}[1]{\langle #1 \rangle}
\newcommand{\xl}{{x_{i-1/2}}}
\newcommand{\xr}{{x_{i+1/2}}}
\newcommand{\il}{{i-1/2}}
\newcommand{\ir}{{i+1/2}}
\newcommand{\keff}{\ensuremath{k_{\text{eff}}}}
\newcommand{\sig}[1]{\ensuremath{\Sigma_{#1}}}


\setlength{\columnsep}{0.5in}

\parskip=0in
\makeatletter
\let\orig@startsection=\@startsection
\let\orig@ssect=\@ssect
\let\orig@sect=\@sect
\newskip\orig@parskip
\orig@parskip\parskip % just for safety's sake!

\def\@startsection{%
 \orig@parskip\parskip%
 \orig@startsection}

\def\@ssect#1#2#3#4#5{%
 \orig@ssect{#1}{#2}{#3}{#4}{#5}%
 \parskip\orig@parskip}

\def\@sect#1#2#3#4#5#6[#7]#8{%
 \orig@sect{#1}{#2}{#3}{#4}{#5}{#6}[#7]{#8}%
 \parskip\orig@parskip}

\makeatother

\setlength{\parindent}{0.25in}

\setlength{\bibsep}{0.0pt}

%-----------------------------------------------------------------------------------------
\title{\boldmath A High-Order Low-Order Algorithm with
Exponentially-Convergent Monte Carlo for $k$-Eigenvalue problems}
\author{S.R. Bolding\inst{1} and J.E. Morel\inst{2}}
\institute{
    \inst{1}\textit{Texas A\&M University Nuclear Engineering Department,
    \href{mailto:sbolding@tamu.edu}{sbolding@tamu.edu}} \\
    \inst{2}\textit{Texas A\&M University Nuclear Engineering Department,
    \href{mailto:morel@tamu.edu}{morel@tamu.edu}}}
\begin{document}

\twocolumn[
\maketitle
\vspace{-0.204in}]
\pagestyle{empty}
\thispagestyle{empty}

\section*{INTRODUCTION}

Moment-based hybrid Monte Carlo (MC) methods have been proven useful for accelerating
solutions to slowly-convergent, iterative particle transport problems.
\begin{comment}
requiring fixed-point iterations (e.g.,
eigenvalue or source iteration problmes).
\end{comment}
Recent work has focused on fixed-point iteration high-order low-order
(HOLO) approaches~\cite{willert}.   Such methods utilize a low-order (LO) operator based on angular moments of the
transport equation, formulated over a coarse spatial mesh. The LO system is typically
based on non-linear diffusion acceleration.  The high-order (HO) system is typically
defined by the transport equation of interest.  Physics operators that are time
consuming for the HO solver to resolve, e.g., fission and
scattering, are moved to the LO system.  The lower-rank LO equations can be solved
efficiently.  Newton-Krylov methods allow for non-linearities in the LO equations to
be efficiently resolved~\cite{willert}.

High-fidelity solutions can be achieved with HOLO methods by using standard
MC simulations to solve the HO transport
equation. The HO solution is used to construct consistency
terms that require the LO solution to be consistent with the HO solution.  These
consistency terms preserve
the accuracy of the MC solution method in the LO operator. Such HOLO algorithms suffer from stability issues caused by statistical noise introduced into the
consistency terms estimated via MC simulation.

In this work, we demonstrate the utility of a unique LO operator in conjunction with
an exponentially-convergent Monte
Carlo (ECMC) method\cite{jake} for the HO solver.  The ECMC algorithm
allows for statistical noise to be reduced to the same order as the HOLO iteration error. We have derived the LO operator
directly from the transport equation, using a linear-discontinuous (LD)
finite-element (FE) spatial
discretization, such that the
HO and LO solutions are consistent upon convergence.  Herein we describe the HOLO algorithm in detail.  We give a brief description of the HO and
LO solvers, without formal derivations
for brevity.  We present results for one-dimensional,
one-speed, $k$-eigenvalue problems. 

\section*{ACKNOWLEDGEMENTS}

This research was performed using funding received from the DOE Office of Nuclear
Energy's Nuclear Energy University Programs.

\section*{METHODOLOGY}

The 1D transport equation to be solved is
\begin{equation}
    \mu \pderiv{\psi(x,\mu)}{x} + \Sigma_t \psi(x,\mu) = \frac{\Sigma_s \phi(x)}{4 \pi} +
    \frac{1}{\keff} \frac{\nu \sig f \phi(x)}{ 4 \pi},\label{ho_trans}
\end{equation}
where $\psi(x,\mu)$ is the angular flux (n/cm$^2$-s) as a function of position $x$
(cm) and $x$-direction cosine $\mu$, $\nu$ is the mean number of neutrons produced
per fission, and $\Sigma_f$, $\Sigma_s$, and $\Sigma_t$ are the macroscopic fission, 
scattering, and total cross sections (cm$^{-1}$), respectively.  
  Eq.~\eqref{ho_trans} represents an eigenvalue
problem for the desired fundamental eigenvalue $\keff$ and eigenmode $\nu \Sigma_f \phi$,
where $\phi=2\pi \int_{-1}^1 \psi(x,\mu) \d \mu$ is the usual scalar flux. This
equation can be solved with standard MC power iteration, but serious difficulties are
encoutered for high dominance ratio systems~\cite{willert}.  As an alternative, we form
a fixed-point iteration between a HO ECMC solver and a LO iterative eigenvalue
solver.  

In the HOLO context, the LO solver handles all of the scattering and eigenvalue
iterations.  The LO equations are similar to double-$P_0$ equations, only with
spatially-dependent angular consistency parameters that are analogous to a variable
Eddington factor.  These consistency parameters are lagged in each LO solve,
estimated from the previous HO solve.
If the angular consistency parameters were estimated exactly, then the LO equations are exact with respect to the chosen
spatial discretization.  

The LO solution is used to construct a spatially LD representation of the right
hand side of Eq.~\eqref{ho_trans}, defined as  $q$ for later reference.  This defines a fixed-source, pure absorber
transport problem for the HO operator.  The HO transport problem is solved using ECMC with adaptive mesh
refinement.  The HO problem defines a characteristic method that uses MC to
invert the continuous streaming plus removal operator with an LD representation of $q$.  The ECMC algorithm allows for the statistical noise in the MC
solution to be reduced to any desired precision within the limits of computational memory and
roundoff.  Thus, the HO solve produces a
globally-accurate, LD representation of the angular flux
$\tilde{\psi}(x,\mu)$.  Here, we emphasize that the ECMC algorithm computes a
projection of the exact solution onto a LDFE space-angle trial space, rather than a
standard FE solution.  This is in general far
more accurate than a standard FE solution.  Once computed, the projected
angular flux $\tilde{\psi}(x,\mu)$ is used to evaluate the LO
consistency parameters for the next LO solve.  

The process of a LO estimate of the transport sources $q$, followed by a HO
solve to estimate LO parameters, represents one HOLO fixed-point iteration.
The consistency terms force the HO
and LO solutions for $\phi(x)$ to be consistent to the order of the current HOLO
iteration error.  Iterations are repeated until convergence.  Currently, the HOLO convergence criteria is based on
$e_{src}^{m} = \|\phi^{m} -
\phi^{m-1}\|_2/\|\phi^{m} \|_2$, the $L_2$ relative change of the point-wise
difference in $\phi$ of successive LO solves in iterations $m$ .
The HOLO algorithm is
\begin{enumerate}
\item Perform a diffusion-equivalent solve of the LO system to
  produce $\phi^{0}(x)$ and $\keff^0$.
\item Normalize $\phi^{m}$.
\item Solve the HO system for $\tilde\psi^{m+1/2}$ using ECMC, based on the current
    LO estimate of the fixed-source $q^m$.
\item Compute LO consistency parameters with $\tilde{\psi}^{m+1/2}$.  
\item Solve the LO system using HO consistency parameters from $m+1/2$ to produce
    $\phi^{m+1}$ and $k^{m+1}$.
\item Repeat 2 -- 5 until $e_{src}^{m}$ is less than a desired tolerance.
\end{enumerate}

\subsection*{The LO System}
\label{sec:lo}

The LO system is formed by taking spatial basis moments and half-range angular
integrals 
of Eq.~\eqref{ho_trans}.  The spatial moments are taken over each spatial cell $i$:
$x\in[x_{i-1/2},x_{i+1/2}]$, weighted with the standard linear Lagrange
interpolatory basis functions.  For example, the left moment operator is defined by
\begin{equation}\label{x_mom}
\mom{\cdot}_{L,i} = \frac{2}{h_i} \int_{x_{i-1/2}}^{\xr} b_L(x) (\cdot) \d x,
\end{equation}
where $h_i=x_{i+1/2}-x_{i-1/2}$ is the width of the spatial element and
$b_L(x)=(x_{i+1/2}-x)/h_i$ is the FE basis function corresponding to position
$x_{i-1/2}$. The positive and negative half-range integrals of the angular flux are defined as
$ \phi^\pm(x) = \pm2\pi \int_0^{\pm 1} \psi(x,\mu) \d \mu$.
Thus, in terms of half-range quantities, $\phi(x) = \phi^-(x) + \phi ^+(x)$.  

Pairwise application of the $L$ and $R$ (right) basis
moments with the $+$ and $-$ half-range integrals to the transport equation
ultimately yields four
equations, for each cell.  Due to space limitations, only the final form of one
equation is given for reference as Eq.~\eqref{lo_eq} in the appendix.  The four
degrees of freedom (DOF) over each cell $i$ are $\mom{\phi}_{L,i}^+$,
$\mom{\phi}_{R,i}^+$, $\mom{\phi}_{L,i}^-$, and $\mom{\phi}_{R,i}^-$.  
The angular consistency terms are defined by the half-range averages
\begin{equation}\label{const}
\mom{{\mu}}_{L,i}^+ =  \frac{
{\displaystyle \frac{2}{h_i}} \int\limits_0^1 \int\limits_\xl^\xr \mu \, b_L(x)
\psi(x,\mu) \d x \d \mu } 
{{\displaystyle \frac{2}{h_i}} \int\limits_0^1 \int\limits_\xl^\xr \, b_L(x)
\psi(x,\mu) \d x \d \mu } .
\end{equation}
The consistency terms for the $R$ basis moment and $\mu_{i\pm1/2}^\pm$ face
terms are defined similarly.


The four coupled equations over each cell for the four DOF are exact; they have not been discretized
in any way.  However, the consistency parameters (defined by Eq.~\eqref{const}) are not known a priori, and
there is no relation between the volume and face averaged quantities.  In the HOLO
context, the equations for unknowns at iteration $m+1$ use consistency parameters
computed using Eq.~\eqref{const} and the latest HO solution $\tilde{\psi}^{m+1/2}$. For the initial LO
solve, all average $\mu$ parameters are
set to $\pm 1/\sqrt{3}$, yielding a solution equivalent to diffusion
with Mark boundary conditions (i.e., an S$_2$ solution).  To close the LO system spatially, the usual LD upwinding
approximation is used, which is consistent with the HO solution.  For example, for Eq.~\eqref{lo_eq}, the face terms $\mu_{i-1/2}$ and $\phi_{i-1/2}$
terms are upwinded from the previous cell $i-1$ (or boundary condition); the terms
at $x_{i+1/2}$ are linearly extrapolated, computed using the $L$ and $R$ basis moments.
  

Summing the equations over all cells, the global LO eigenvalue problem is formed.  In
operator notation, the global system is 
\begin{equation}\label{lo_k}
    \B D \Phi = \frac{1}{\keff} \B F \Phi,
\end{equation}
where $\Phi$ is the vector of unknown space-angle moments, $\B D$ represents
streaming plus removal minus scattering, and $\B F \Phi$ is the
fission source.  The matrix $\B D$ is an asymmetric banded matrix with a bandwith of
seven, which can
be directly inverted efficiently
without need for inner source iterations.  Eq.~\eqref{lo_k} is solved using inverse
power iteration, i.e., given $\keff^{(0)}$ and $\Phi^{(0)}$,
\begin{align*}
    \Phi^{(l+1)} = \frac{1}{\keff^{(l)}} \B D^{-1} \B F \Phi^{(l)}, &
    \quad \keff^{(l+1)} = \keff^{(l)}\frac{ \int \nu \sig f \phi^{(l+1)} \d
    x}{ \int \nu \sig f \phi^{(l)}\d x }.
\end{align*}
The $l$ iteration indices here are independent of the HOLO iteration $m$.
The above power iteration is repeated until $\keff^{(l)}$ and $\B F \Phi^{(l)}$ are
converged.  


\subsection*{The ECMC High Order Solver}

The transport equation to be solved by ECMC is given by Eq.~\eqref{ho_trans}. 
For the HOLO context, in operator notation, this equation is defined as
\begin{equation}\label{te_oper}
\B L \psi^{m+1/2}  = q^{m}
\end{equation}
where $\psi^{m+1/2}$ is the solution of the angular flux based on the $m$-th
LO estimate of $q$.  The linear operator $\B L$ is the streaming plus
removal operator defined by the left
side of Eq.~\eqref{ho_trans}.  The fixed
source $q^{m}$ is taken to lie in the LD space-angle trial space that will be used to represent
$\psi$.  The HO solver uses a space-angle mesh, with each cell $(i,j)$
defined as the region of phase space $x \in [x_{i-1/2},x_{i+1/2}] \times \mu \in
[\mu_{j-1/2}, \mu_{j+1/2}]$, where $(x_i,\mu_j)$ is the center of the cell.

To define the ECMC algorithm, the HOLO iteration indices
$m$ are dropped, noting that the LO estimated $q^{m}$ remains constant over the entire HO solve.
The $i$-th approximate solution to Eq.~\eqref{te_oper} is $\tilde{\psi}^{(i)}$, where
$i$ identifies the MC batch.
The $i$-th residual is $r^{(i)} = q - \B L\tilde{\psi}^{(i)}$, which with manipulation gives the error equation
\begin{equation}
\B L (\psi - \tilde{\psi}^{(i)}) = \B L \tilde{\epsilon}^{(i)} = r^{(i)}
\end{equation}
where $\psi$ is the exact solution and $\tilde{\epsilon}^{(i)}$ is finite element
representation, in space and angle, of the error in
$\tilde{\psi}^{(i)}$. The above equation represents a second, fixed-source, pure
absorber transport equation.
The operator $\B L$ is inverted without discretization via MC to produce an
estimate of the error in $\tilde{\psi}^{(i)}$, i.e., $\tilde{\epsilon}^{(i)} = \B
L^{-1} r^{(i)}$.
The projections of $\epsilon$ and $\psi$ onto the LD space-angle trial space are computed using standard
volumetric path-length
estimators.  The estimators are weighted by appropriate basis functions 
to tally the zeroth and first moments, in $x$ and $\mu$, of 
$\psi(x,\mu)$ over each space-angle cell.
  The ECMC algorithm is
\begin{enumerate}
\item Solve Eq.~\eqref{te_oper} to compute the MC projection of the angular
flux onto the LD $x-\mu$ trial space $\tilde\psi^{(0)}$.
\item Compute $r^{(i)}$.
\item Solve $\tilde{\epsilon}^{(i)} = \B L^{-1} r^{(i)}$
\item Compute $\tilde\psi^{(i+1)} = \tilde\psi^{(i)}
+ \tilde\epsilon^{(i)}$
\item Repeat 2 -- 4 until desired convergence of the angular flux is achieved.
\end{enumerate}

Convergence is based on the estimated relative error in $\tilde \psi^{(i)}$, defined as
$\epsilon_{rel} = \| \tilde \epsilon^{(i)} \|_2/\| \psi^{(0)} \|_2$.
Exponential convergence (with respect to the number
of particles simulated) of $\tilde \epsilon$ is obtained because with each inversion of $\B L$ a
more accurate estimate of the solution is used to compute the new residual, decreasing
the magnitude of the MC source $r^{(i)}$.  If the statistical estimate of $\tilde\epsilon$ is not sufficiently
accurate, then the batches would diverge.
Because the true angular flux does not in general lie within the LD trial space, the
iterative estimates of the error will eventually stagnate.  An adaptive $h-$refinement algorithm is used
to allow the system to continue converging towards the exact solution. 
Because some regions of refined cells are relatively small, the probability of particles being
sampled from those cells is very low.  To resolve issues this causes, stratified sampling is
used to sample the same number of source particles from each space-angle cell.  
Currently, the number of particles sampled
in each cell remains constant throughout the HO solve.


\section*{RESULTS}

\begin{comment}
    To verify the HOLO algorithm, the method of manufactured solutions was applied for
fixed source problems, as in ~\cite{jake}.
\end{comment}

We have simulated two 1-D, 1-speed test problems to demonstrate the utility of
this HOLO algorithm for eigenvalue problems.  The first, problem
A, is an analytic benchmark from~\cite{sood}.  The problem represents a
bare, critical slab of Pu$^{239}$ with a critical width of 4.513502 cm.  Problem B is a nearly-critical, optically thick slab with a 100 cm
width. The one-group
nuclear data for both problems are given in
Table~\ref{problems}.  For both problems, the spatial
mesh consisted of 100 equal-sized
cells.  The initial HO mesh at each $m$ has a width of 100
cells in $x$ and 20 angular cells in $\mu$.     Table~\ref{k1} presents numerical
results.the $m=0$-th HOLO iteration is the initial diffusion-equivalent
solve of the LO system.   Subsequent LO solves are initialized with $\phi(x)$ from the previous
HOLO iteration.   

\subsection*{Problem A}

As in~\cite{willert}, a dynamic convergence was used for the HO solver in problem A to
efficiently distribute the number of particle histories amongst the HOLO iterations.
Starting from some fixed tolerance, the HO convergence tolerance is reduced as
$e_{src}^{m}$ decreases with each HOLO iteration.  The choice of HO convergence
criteria was chosen to be $\epsilon_{rel}^{m+1/2} < \min \,
\{e_{src}^m/4,\; 0.01\} $.  The minimum HO and $e_{src}$ convergence
criterion were 1.E-04, and an iterative relative error criteria on the LO solver of
1.E-06 for the fission source and \keff.  

With $2.4$E$+10^7$ histories, the
eigenvalue was determined accurately to below the desired precision of 1E-04. We repeated this
simulation using a total of ten different random number seeds, resulting in an
average of $\keff = 1.0000026$ and standard deviation 3.1E-06.  The maximum
deviation of $\keff$ from all simulations was below 1.E-05. The fission source was
found to agree, within expected spatial truncation error, with data points given
in~\cite{sood}.
 Additionally, even though only
100 spatial cells are used, because the consistency parameters ensure the LO
equations are accurate to the tolerance of the HO solver, an accurate solution is still obtained.  The dominate
truncation error becomes the LD representation of $q$ in the ECMC algorithm.   

As an estimate of the
efficiency of ECMC in the HOLO context, we estimated the number of histories that would be needed
using standard MC to achieve the same error reduction over each HO solve.  Based on
a $1/\sqrt{N}$ reduction in stochastic error, 
the number of standard MC histories needed is estimated as
$N^{(0)}(\epsilon_{rel}^{(0)}/\epsilon^{(N_B-1)}_{rel})^2$, where $N_B$ is the total
number of batches.  For this problem, a total
of 6.9E+09 histories would be needed to achieve the same error reduction, compared to
$2.4$E$+07$ for ECMC.  

\subsection*{Problem B}

The second problem has a difficult fission source to converge due to the large optical
thickness and scattering ratio ($\sig s/ \sig t$).  The dominance
ratio of this
problem was numerically
estimated to be $0.9993$.  It was computed based on the
asymptotic iterative convergence of the fission source in the initial LO
diffusion-equivalent solve.  Dynamic convergence tolerance was not used for the HO solver because of convergence
issues with the LO fission source.  The LO and HOLO convergence criteria were 1.E-04,
with a HO convergence criteria of 1.0E-05.

Because of the high scattering and dominance ratio, the fission
source in this problem would be very difficult to converge with standard Monte
Carlo power iteration, requiring 995 LO power iterations.
The LO operator can determine the distribution of the fission source much more
efficiently, noting the performance of the HOLO algorithm would be improved significantly by using a Newton-Krylov
method to solve the LO eigensystem as in~\cite{willert}; this would also mitigate
false source convergence in high dominance ratio problems.  The HO consistency terms
added minimal correction.  However, as this problem demonstrates the HO system can be solved to a low tolerance with relatively few mesh refinements
and particle histories because the LD representation of the flux is accurate for this very
diffuse problem.  

\section*{CONCLUSIONS}

We have shown that we can compute the fundamental eigenvalue and eigenmode using our
HOLO algorithm.  The LO solver handles the eigenvalue solves and is derived such that
both operators are converging to the same solution.  ECMC allows for the angular flux
to be accurately resolved to the order of the error of the outer fixed-point
iterations significantly more efficiently that standard MC.  Also, ECMC is very effective in the HOLO context where
particles are distributed throughout the problem by the LO solver, eliminating issues with optically
thick regions.  The HOLO algorithm is desirable in situations where globally accurate
solutions are needed over a domain with a mix of diffuse and absorptive regions.
Future work will include extending to multi-group equations and multiple spatial
dimensions 

\bibliography{references}
\bibliographystyle{ans}    

\setlength{\tabcolsep}{3pt}
\begin{table}[ht!] \center
    \caption{Problem parameters and identifiers; all cross sections are in cm$^-1$,
        $\Sigma_c$ is the capture cross. \label{problems}}
  \begin{center}
  \hspace{-0.1in}
  \begin{tabular}{cccccc}
       \hline Problem & $\nu$ & $\sig f$ & $\sig c$ & $\sig s$ &  \sig t  \\
       \hline  A & 2.84 & 0.0816 & 0.019584 & 0.225216 & 0.32640  \\
               B & 2.00 & 0.25     & 0.25     & 29.5     & 30.0      \\
       \hline  
  \end{tabular}
  \end{center}
\end{table}
\setlength{\tabcolsep}{5pt}
\begin{table}[ht!] \center
    \caption{Results for two test problems: LO iteration counts $I$, total number of
        particle histories $N$, and MC batches, for HOLO iterations $m$.
 \label{k1}}
  \vspace{0.15in}
  \begin{center}
  \hspace{-0.1151in}
  \begin{tabular}{|c|cc|c|cc|}
      \hline \multirow{1}{*}[-0.24em]{HOLO} & \multicolumn{2}{|c|}{HO} &
      \multicolumn{1}{|c|}{LO} & \multirow{2}{*}{\keff} & \multirow{2}{*}{$e_{src}^{m}$} \\
      \cline{2-4} \multirow{1}{*}[0.23em]{ Iter. $m$} & $N_B$ & $N$ & $I$ & & \\ \hline \hline
      \multicolumn{6}{|c|}{\textbf{Problem A}} \\ \hline
      0 & - & -         &7 &    0.835654 & - \\
      1 & 3 & 7.2E+05   &7 &    1.000015 & 4.0E-02  \\
      2 & 4  & 9.6E+05  &6 &   0.998877 & 2.9E-03 \\
      3 & 17 & 10.0E+06 &7 &  1.000003 & 5.4E-04 \\
      4 & 18 & 11.9E+06 &4 &  1.000001 & 5.0E-05 \\ \hline \hline
      \multicolumn{6}{|c|}{\textbf{Problem B}} \\ \hline
      0 & - & -         & 995 &    0.999939 & - \\
      1 & 38 & 6.8E+06  & 2   &   0.999941 & 1.0E-04  \\
      2 & 39 & 7.4E+06  & 1   &   0.999940 & 9.9E-05 \\ \hline
  \end{tabular}
  \end{center}
\end{table}
\section*{APPENDIX}
\appendix 

The LO equation for positive flow and the left FE basis moment is given by
\vspace{-0.02in}

\begin{multline}\small 
\label{lo_eq}
    -2{\mu}_{i-1/2}^+ \phi_{i-1/2}^+ + \mom {\mu}_{L,i}^+ \mom{\phi}_{L,i}^+ +
    \mom\mu_{R,i}^+ 
    \mom{\phi}_{R,i}^+ + \Sigma_t h_i \mom{\phi}_{L,i}^+ \\ = h_i \left(\Sigma_s + 
    \frac{\sig f \nu}{\keff} \right)   \left( \mom{\phi}_{L,i}^+ +
\mom\phi_{L,i}^-\right) 
\end{multline}





\end{document}
